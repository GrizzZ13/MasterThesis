% !TEX root = ../main.tex

\chapter{\sysname 系统架构与设计}

本章将介绍 \sysname 的整体设计理念、系统结构以及核心组件的工作方式。为了应对大模型推理过程中点对点通信效率不足的问题,\sysname 在系统设计上充分利用 GPU 集群的横向 RDMA 网络与纵向 NVLink 高速互联,构建了一个在不占用 GPU 计算核心的前提下,支持高带宽、低时延传输的通信库。

\section{系统概述}

\sysname 是面向大模型推理场景设计的高性能通信库,其目标是在推理过程中提供高效、稳定、可扩展的节点间数据传输能力。与传统的 GPU 通信方式相比,\sysname 采用控制面完全卸载(fully control-plane offloading)的方式,将通信调度工作交由 CPU 侧的运行时线程池处理,避免占用 GPU 计算资源,使得推理任务能够最大化地利用 GPU 的算力。\sysname 的设计主要具有以下特点:

\begin{itemize}
    \item 异构多链路支持。 \sysname 采用统一的双向通信接口抽象,用户侧仅需通过 \texttt{send()} 与 \texttt{recv()} 两个接口即可完成通信调用。系统内部能够根据硬件环境自动选择使用 GPU Direct RDMA 或 NVLink 作为底层链路,从而在不同网络拓扑下提供统一的编程接口,屏蔽底层差异性简化开发复杂度。
    \item 良好的可扩展性。 \sysname 采用分层化架构,上层接口稳定,底层链路可插拔。GPU Direct RDMA 和 NVLink 的实现遵循同一调度和接口语义,通过替换链路执行器即可无缝接入更多未来通信方式。
    \item 安全性与高可靠性。系统基于现代 C++ 与 RAII 原则实现,自动管理资源生命周期,减少手动管理带来的风险。为了在高并发场景下保持性能,系统内部广泛采用无锁队列与细粒度的并发控制机制,确保在高负载下仍能保持正确性与稳定性。
\end{itemize}

图 \todo{} 展示了 \sysname 的整体系统架构。系统由两部分组成:应用接口层 与 运行时执行层。应用接口层负责接收上层任务,构建通信指令,并将其通过无锁队列传给后台运行时执行层进行处理,实现用户线程与通信线程的解耦。
运行时执行层由绑定线程池驱动,每个线程通过轮询(poll)机制获取任务并完成通信调度,包括 RDMA 收发指令的组织、WR 列表构建、发送与完成队列处理等。当传输任务完成时,运行时触发事件回调(event notify),唤醒等待中的用户侧线程,从而实现异步高效的通信处理流程。

这种设计将通信调度与 GPU 核心计算资源彻底隔离,实现了对模型推理任务最小的额外资源占用,同时保证了通信通路的可扩展性和高性能表现。

\section{分层架构设计}

为了在多样化的 GPU 集群硬件环境中实现可扩展的通信能力,\sysname 采用了严格的分层架构设计。该架构主要由三层组成:应用接口层、运行时调度层与链路执行层。通过分层化抽象,系统实现了接口与硬件细节的解耦,使得不同通信链路(如 RDMA、NVLink)可以在保持接口语义一致的前提下独立演进。

\subsection{应用接口层}

应用接口层是 \sysname 暴露给用户侧的最上层抽象,其目标是提供统一、轻量且对硬件无感知的通信接口。无论底层使用 RDMA 还是 NVLink,用户始终通过 \texttt{send()} 与 \texttt{recv()} 两个操作完成数据传输请求。该层负责将用户的调用参数封装为通信任务(communication task),并通过无锁队列提交到运行时调度层,从而实现用户线程与通信线程的完全解耦。

同时通过事件机制(Event),应用接口层支持异步通信模型。用户在调用 \texttt{send()} 或 \texttt{recv()} 后,立即获得一个事件对象(Event),该对象可用于查询传输状态或等待传输完成,极大提升了通信的并发性与灵活性。事件对象由 \sysname 系统统一创建与管理,并在底层数据传输完成后由运行时调度层负责触发唤醒,实现了高效的异步通信流程。

\subsection{运行时调度层}

运行时调度层是 \sysname 的核心控制面逻辑所在。该层采用 CPU 线程池负责执行通信调度和轮询,由 CPU 线程驱动通信任务的执行不占用 GPU 流式多处理器,实现完全控制面卸载。其主要功能包括:

\begin{enumerate}
    \item 接收来自应用接口层的任务,并将其入队至内部待发送任务队列;
    \item 根据来自接收者(Receiver)的请求,进行通信任务匹配,并将匹配成功的任务入队至发送队列;
    \item 根据任务优先级选择发送队列中的任务进行调度,并通过合适的链路执行器发起数据传输;
    \item 通过轮询(poll)机制监控链路状态,等待传输完成;
    \item 在任务完成时修改事件状态,触发通知唤醒等待中的用户线程。
\end{enumerate}

该层的设计保证了即使在大型集群的高并发场景下,系统仍能保持低调度开销和高吞吐能力。

\subsection{链路执行层}

链路执行层封装了各类通信链路的实现逻辑,其行为由运行时调度层驱动,不对用户可见。RDMA、NVLink 等不同执行器具有各自的队列结构与传输语义,但统一遵循应用接口抽象与调度层规范。

这一层的可插拔设计保证了 \sysname 可以随着未来 GPU 拓扑与互联技术的发展而平滑扩展,而无需对上层接口和调度逻辑作任何修改。

\section{应用接口层设计}

应用接口层是 \sysname 面向用户的入口,其设计目标是:保持接口简洁、强语义一致性、对底层链路透明,同时具备异步化和可扩展能力。

\sysname 将通信操作统一抽象为 \texttt{send()} 与 \texttt{recv()} 两类基本指令。每一次指令调用均会生成一个独立的通信任务,系统负责保证以下语义:

\begin{itemize}
    \item 双向通信的接口形式保持一致;
    \item 用户完全无感知底层链路差异;
    \item 不强制绑定 CUDA Stream,也不占用 GPU 计算资源;
    \item 传输完成的通知由事件机制统一管理。
\end{itemize}

在用户调用 \texttt{send()} 与 \texttt{recv()} 向系统提交通信任务时,每个通信任务均携带一个唯一标识符 \texttt{unique\_id},其在 \sysname 中承担两类功能:

\begin{enumerate}
    \item \textbf{作为逻辑通道唯一标识}:\texttt{unique\_id} 用于标识发送端与接收端之间的逻辑通信通道,确保数据能够正确路由到对应的接收者,以支持同一时刻进行多路并发通信,达到物理链路的多路复用,提升整体资源利用率。
    \item \textbf{优先级控制}:运行时调度层以 \texttt{unique\_id} 为最基本的公平性与优先级调度单元,通过维护不同优先级的任务队列,实现基于优先级的动态带宽分配。高优先级的 \texttt{unique\_id} 可获得更多调度机会,从而降低其传输延迟,满足延迟敏感型通信需求。
\end{enumerate}

通过增加带有优先级语义的唯一标识符,系统可以为多级优先级调度提供了接口层基础,并且可扩展至更多调度策略,提高可扩展性。

\sysname 采用显式事件(Event)对象实现通信异步化。每次 \texttt{send()} 或 \texttt{recv()} 返回一个 \texttt{Event},其语义如下:

\begin{itemize}
    \item Event 初始状态为未完成;
    \item 当传输在运行时调度层完成后,链路执行器触发事件;
    \item 用户可主动调用 \texttt{wait()} 阻塞等待,或调用 \texttt{is\_notified()} 进行事件状态查询。
\end{itemize}

这一机制避免了传统接口依赖 Cuda Event 和 Stream 的限制,使得通信流程完全独立于 GPU 计算流水线。

为确保低延迟与高并发,应用接口层采用多生产者、多消费者的并发队列(Multi-Producer Multi-Consumer Concurrent Queue)将任务提交给运行时调度层。任务封装内容包括:

\begin{itemize}
    \item 通信元数据:地址、长度、访问控制的密钥信息等;
    \item 关联的 \texttt{unique\_id} 信息;
    \item 对应的事件对象,用于后续运行时调度层在传输完成之后通知用户接口层。
\end{itemize}

通过这种方式,用户线程不会因提交通信任务而发生阻塞,从而使上层推理任务能够以最高并行度驱动模型流水线。

\section{RDMA 链路执行器设计}

RDMA(Remote Direct Memory Access)链路执行器是 \sysname 的核心数据面组件之一,其负责在 GPU Direct RDMA 网络上完成端到端的数据传输。该执行器遵循统一的链路抽象规范,被运行时调度层(Runtime Scheduler)驱动,实现无需 GPU SM 参与的高性能数据搬运。

本节将详细介绍 RDMA 执行器的内部机制,包括内存注册、控制面与数据面的协同方式、基于 Ticket 的接收方发起机制,并详细解释 \sysname 为了保证多 GPU 环境下通信的系统性能和正确性做出的针对性优化。

\subsection{GPU 内存注册与 MemoryRegion 管理}

RDMA 网卡 RNIC 只能通过直接内存访问(Direct Memory Access)访问到已经注册的内存区域,因此 RDMA 链路执行器需要在执行任何传输之前,将数据缓冲区注册为 MemoryRegion。MemoryRegion 可以是主机内存,也可以是 GPU 显存。针对 GPU 显存注册的 MemoryRegion 可以通过 nvidia\_peermem 内核模块在进行 DMA 访问时进行地址翻译。在注册 MemoryRegion 时,通过调用 libibverbs 的接口进行注册,注册得到的 MemoryRegion 可以获取用于本地访问权限控制的密钥 lkey 和用于远端访问权限控制的密钥 rkey,保证数据传输的安全性。

RDMA 网卡(RNIC)在执行远程读写操作时,并不能像 CPU 那样通过多级页表和缺页异常机制对任意虚拟地址进行透明访问。网卡自身维护着一套设备内存翻译表(Memory Translation Table, MTT),其中保存了系统页表的一部分内容。只有在某段内存注册给网卡并且建立了内存翻译表项后,RNIC 才能对其执行 DMA 操作。因此在 RDMA 通信开始之前,任何需要参与 DMA 的主机内存或 GPU 显存都必须通过 \texttt{ibv\_reg\_mr} 注册为 MemoryRegion 才能进行地址翻译。

当对一段主机内存调用 \texttt{ibv\_reg\_mr} 时,libibverbs 会将注册请求交由内核驱动处理。驱动首先会将该虚拟地址区间对应的物理页固定(pin)在内存中,使其不会被操作系统换出(swap-out);随后,驱动会为这些被锁定的物理页建立设备虚拟地址到物理地址的映射,并将映射信息写入 RNIC 内部的地址翻译表,使网卡能够在传输过程中独立完成地址翻译。注册完成后,\texttt{ibv\_reg\_mr} 会返回一个 MemoryRegion,其中包含本地访问所需的 lkey 以及远端节点访问时需要的 rkey,这两个密钥由 RNIC 在硬件中进行权限验证,从而保证了访问的安全性。

然而,GPU 显存在物理组织和寻址方式上均不同于普通主机内存。CUDA 申请的显存地址属于 GPU 设备内部的虚拟地址空间,不存在于 CPU 的线性地址空间中;GPU 页表也由 GPU 驱动维护,操作系统无法直接通过 \texttt{get\_user\_pages()} 固定这些页。因此,为使 RDMA 网卡能够像访问主机内存一样直接 DMA 访问 GPU 显存,系统需要一个能在 RNIC 与 GPU 驱动之间桥接地址翻译与页固定逻辑的内核组件。NVIDIA 提供的 \texttt{nvidia\_peermem} 内核模块基于 Mellanox 的 PeerDirect(即 \texttt{ib\_peer\_memory})框架,将 GPU 显存暴露为一种“外部设备内存”(peer memory)。该模块向 RDMA 协议栈注册为 peer memory client,并提供包括 \texttt{acquire()}、\texttt{get\_pages()}、\texttt{dma\_map()} 与 \texttt{put\_pages()} 在内的一系列回调接口。当应用通过 \texttt{ibv\_reg\_mr} 传入一个由 \texttt{cudaMalloc} 获得的 GPU device pointer 时,RDMA 内核栈会调用各 peer memory client 的匹配函数,最终由 \texttt{nvidia\_peermem} 接管该地址区间的注册流程。
在接管注册之后,\texttt{nvidia\_peermem} 会通过调用 GPU 驱动的 P2P(peer-to-peer)接口来完成 GPU 页的固定与地址翻译过程。具体而言,GPU 驱动负责将对应显存页 pin 住,并返回这些页在 PCIe 体系结构下可被其他设备访问的总线地址描述(通常以 \texttt{sg\_table} 或等价的散列表结构表示)。随后,\texttt{nvidia\_peermem} 将这些页信息回传给 RDMA 子系统,由 HCA 驱动执行对应的 DMA 映射(\texttt{dma\_map\_sg()}),从而获得可直接被网卡访问的 DMA 地址。RNIC 最终使用该地址列表构建自身的 MTT 项,完成 GPU 显存的 MemoryRegion 注册,并向应用返回可用于本地与远端访问权限控制的 \texttt{lkey} 与 \texttt{rkey}。

通过上述机制,\sysname 才能够在 RDMA 数据面上直接操作已经注册的 GPU 内存区域,使 RDMA 链路就在不经过主机内存中转的情况下完成节点间的数据传输,为 \sysname 在异构多链路环境下实现高性能 GPU 通信奠定了基础。

在 \sysname 中,MemoryRegion 工具模块 rdma\_util 在初始化时完成内存注册,并由 Runtime 在每次 send/recv 提交任务时携带 \texttt{addr, length, lkey / rkey} 信息。这些资源在 \sysname 中完全以 RAII 方式管理,避免显式 deregister 带来的复杂性和错误风险。

在注册完成之后,\sysname 利用 GPU Direct RDMA 技术直接完成 GPU-to-GPU 的数据搬移。与传统的 ``$\text{GPU} \rightarrow \text{Host} \rightarrow \text{NIC} \rightarrow \text{Host} \rightarrow \text{GPU}$'' 路径不同,GPU Direct RDMA 将路径缩短为 ``$\text{GPU}_{\text{src}} \longrightarrow \text{NIC}_{\text{src}} \longrightarrow \text{NIC}_{\text{dst}} \longrightarrow \text{GPU}_{\text{dst}}$'',无需主机内存参与数据面搬运,从而提升带宽并降低延迟。

\subsection{Receiver 驱动的双边通信设计}

在 \sysname 中,数据面的传输采用“Receiver 发起(Receiver-Initiated)”模式。其核心是接收方在准备好缓冲区之后,主动告知发送方自己缓冲区的元数据,允许发送方在接收方无感的情况下进行数据写入。接收方告知发送方的信息通过一个轻量级的控制面结构 \textbf{Ticket} 进行传递,其中包括的信息如下:

\begin{itemize}
    \item \texttt{addr}: 接收方(Receiver)内存的起始地址,用于 RDMA write
    \item \texttt{key}: 接收方内存允许对端网卡直接 RDMA write 访问控制的 rkey
    \item \texttt{length}: 传输长度
    \item \texttt{unique\_id}: 用于通道标识与优先级调度
\end{itemize}

在实现上,Receiver 将 Ticket 写入本地提前注册为 MemoryRegion 的 host buffer,然后使用 \texttt{post\_send()} 发送一条双边语义的 RDMA send 将其发送给对端。Sender 端调用 \texttt{post\_recv()} 提前准备足够数量和大小的 Recv WR,从而能够接收 Ticket。这一阶段基于 RDMA 数据面,不涉及操作系统内核的参与,相比于使用带外通信(TCP),保证了高效的控制面通信。发送方获得 \textbf{Ticket} 后,可以进行请求匹配并准备发送队列,根据优先级和通道标识符进行调度,最终构建 RDMA write 请求,在接收方 CPU 无感知的情况下将数据直接写入接收方。

\subsection{Sender 端的匹配与发送队列维护}

具体地,Sender 在每一次调用 \texttt{poll()} 时,都会执行以下步骤:

首先通过 \texttt{post\_recv()} 接收来自 Receiver 的 Ticket,然后通过并发队列接收来自应用接口层的本地待发送任务的 Ticket;其次按照本地和远端接收到的 Ticket 中的 unique\_id 将两类请求匹配;最后构建 RDMA write 请求,通过 \texttt{post\_send()} 提交发送。

其中 Sender 内部维护的并不是单一队列,而是基于 \texttt{unique\_id} 组织的多路独立流控制结构。具体而言,Sender 通过两个 \texttt{MultiMap} 对未匹配的请求进行管理:

\begin{itemize}
    \item \texttt{pending\_remote\_recv\_request\_map}: 以 \texttt{unique\_id} 为键,按逻辑通道存放来自 Receiver 的 Ticket,这些 Ticket 可能包含需要多次分片的剩余区间,因此队列中的元素会在分片过程中被原地更新。
    \item \texttt{pending\_local\_send\_request\_map}: 同样以 \texttt{unique\_id} 为键,按逻辑通道存放本地待发送请求,未被匹配的请求将暂存在该结构中。
\end{itemize}

上述两个映射结构共同构成了 Sender 的“待匹配请求池”。在每个 \texttt{unique\_id} 上,只有当远端请求与本地请求同时存在时,Sender 才能够为该流构造有效的 RDMA Write 操作;否则,相关 Ticket 将继续保留在队列中,等待下一次 \texttt{poll()} 调用完成匹配条件。

一旦两类请求成功匹配,Sender 便根据当前剩余传输大小决定是否需要拆分为多个分片(chunks)。对于每个分片,Sender 构造一个对应的 Work Request 提交给 NIC。最终,在 NIC 返回带有 \texttt{finished} 标志的完成事件(Completion Queue Entry, CQE)后,Sender 才会触发上层事件,表示整个发送过程真正结束。

这种多队列设计通过增加本地缓存,避免了因为并发请求过多或发送方、接收方不同步导致的请求丢失问题,并使得 Sender 能够在面对异步化、高并发的请求流量时保持稳定。进一步地,由于每个请求在完全匹配之前都会被保存在待匹配队列中,Sender 可以灵活地处理大消息拆分,在多优先级任务调度下通过细粒度分片实现近似抢占式的传输控制,避免了队头阻塞和优先级反转问题,从而提升整体通信效率。

\subsection{带有通知机制的 RDMA Write 设计}

在基于 RDMA 的单边写操作的设计中,普通的 \texttt{IBV\_WR\_RDMA\_WRITE} 仅在发送方本地完成队列产生完成事件,而接收方在数据被写入缓冲区之后并不会自动获得通知。如果完全依赖 CPU 轮询 GPU 显存或发送额外的控制消息来感知写入完成,一方面会引入额外的 PCIe 开销,另一方面也会增加控制面的 RTT。为此,\sysname 在 RDMA 链路执行器中采用了带立即数的写操作 \texttt{IBV\_WR\_RDMA\_WRITE\_WITH\_IMM},在不增加额外控制报文的前提下,将数据写入与通知语义合并在同一个 RDMA 操作中完成。

具体地,对于被分片传输的大消息,Sender 端会将中间分片使用普通的 RDMA Write 请求提交,不会在接收方的 Recv CQ 产生时间;只在最后一个分片使用带有立即数的 RDMA Write 请求(操作码为 \texttt{RDMA\_WRITE\_WITH\_IMM})。所有分片共享同一个逻辑请求上下文,通过 Sender 侧维护的 \texttt{wrid\_t} 结构进行状态编码,其中:

\begin{itemize}
    \item \texttt{unique\_id} 标识所属的逻辑通道;
    \item \texttt{doorbell\_length} 记录本次 doorbell 批处理中包含的工作请求数量;
    \item \texttt{finished} 标记当前完成事件是否对应某个逻辑请求的最后一个分片。
\end{itemize}

当 Sender 侧从发送完成队列(Send CQ)中轮询到完成事件后,首先解析 \texttt{wr\_id} 转换成\texttt{wrid\_t} 结构还原上述字段,从而一次性回收 doorbell 批中对应的本地状态;若该完成事件对应的 \texttt{finished} 为真,则说明该 \texttt{unique\_id} 上的整个发送请求已经全部写入完成,此时 Sender 从本地缓存中取出对应逻辑通道的 \texttt{Event} 并调用 \texttt{notify()},向应用接口层交付“发送完成”的通知。

在接收方,\sysname 为每个 Ticket 提前在 Recv CQ 上预投递零长度的接收工作请求,可以用于匹配带有立即数(Immediate)的 RDMA Write 请求。当 Sender 使用带有立即数的 RDMA Write 完成最后一个分片写入时,RNIC 在执行写入的同时,会在接收者的 Recv CQ 上生成一条带有 \texttt{IBV\_WC\_RECV\_RDMA\_WITH\_IMM} 操作码的完成事件。该事件包含一段 32 位的立即数,\sysname 将其编码为 \texttt{unique\_id},Receiver 在取出该事件后,即可直接索引到对应的逻辑通道,并进一步通知应用接口层传输完成的语义。

\subsection{GPU 内存一致性问题}

在 GPU Direct RDMA 场景下,RDMA 网卡(RNIC)在完成对 GPU 显存(device memory)的 DMA 写入后,其写入顺序和可见性并不能仅由 RDMA Completion 事件来保证。根本原因在于 PCIe 协议允许对不同目标地址的写操作发生乱序(out-of-order),从而导致这样一种情况:接收端虽然已经在 RNIC 的完成队列中观察到 \texttt{IBV\_WC\_RECV\_RDMA\_WITH\_IMM} 的完成事件,但 GPU 上对应的数据可能尚未在内存体系结构中变为可见。因此,RDMA completion 所提供的仅是写入已由 RNIC 执行完毕的语义,而不是 GPU 能够读取到最新写入数据的可见性语义。

为确保 GPU 能以强一致性观察到由 RNIC 写入的数据,\sysname 在接收路径中引入了独立的 \texttt{Flusher} 组件。当系统处于 GPU Direct RDMA 模式时,接收方在捕获到 \texttt{IBV\_WC\_RECV\_RDMA\_WITH\_IMM} 的完成事件之后不会立即向上层触发事件通知,而是将该事件交由 \texttt{Flusher} 处理。具体而言,\texttt{Flusher} 会对刚被写入的 GPU 地址执行一次本地回环(loopback) 的 RDMA read 操作。由于 PCIe 协议保证:对于同一事务域中来自同一发起方的 PCIe 写,在该写之后发起的读操作必须能够观测到之前所有写入结果,因此这次 loopback read 可以作为一道内存栅栏(memory fence),确保此前 RNIC 对 GPU 显存进行的所有 DMA 写都已在 GPU 内存层次结构中完成并对后续访问可见。

通过这种机制,\sysname 获得了强于原生 RDMA completion 的可见性与排序语义。上层应用在收到来自 \texttt{Flusher} 的事件通知时,可以严格保证 GPU 侧数据已完成写入并可见的语义,从而避免由于 PCIe 写入乱序带来的数据不一致问题。

\subsection{RDMA 性能优化}

为了让 \sysname 在大规模 GPU-to-GPU 数据传输和高并发场景下,真正发挥出 RDMA 网络高带宽与低时延的能力,RDMA 执行器在实现层做了多项优化,以尽可能减少 CPU \& RNIC 交互、降低 PCIe 总线与 NIC 负担。

首先,\sysname 利用 WR list + doorbell batching 方式,将多个独立的工作请求(WR)组织成一个链表,然后一次性通过 \texttt{ibv\_post\_send()} 提交给 RNIC。这样,多个 WR 对应的通知 (doorbell) 只需一次 MMIO 写 (doorbell ring),而不是为每个 WR 都做一次。对比传统逐 WR 提交方式,这样能显著减少 CPU 发起 MMIO 的次数,从而降低 CPU-NIC 的交互开销。

其次,\sysname 也进行了对 Work Completion 的优化,进行选择性的设置 Signal WR,并与批量 WR 提交结合,使得中间的分片 (chunk) WR 不产生 CQE(Completion Queue Entry),从而避免 RNIC 对每个 WR 都通过 PCIe DMA 写回主机,大幅减轻 NIC 和 PCIe 总线写回负担。仅在 WR 链表满或是逻辑消息的最后一个 WR 设置为触发 Signal,以保证发送方能够及时获知传输完成状态,并进行批量的 WR 链表缓冲区复用。这样不仅保证了语义正确,也最大限度地降低了开销。

在 \sysname 内部,这些优化与前面提到的分片(chunking)、带立即数的 RDMA write + notify、以及用于 GPU 显存一致性的 \texttt{Flusher} 机制配合。具体来说,当发送大消息时,系统会将消息拆为多个分片,对这些分片构建多个 WR 并以链表形式批量提交,中间 WR 标记为 unsignaled,仅最后一个 WR 标记为 signaled + with IMM。RNIC 会一次拉入整个 WR 链 (DMA read),然后执行分片写操作,当最后一个写完成时触发 CQE,并通过带有立即数的 write\_with\_imm 告知接收端。接收端再启动 \texttt{Flusher} 做 read fence,确保 GPU 可见性后才通知上层,从而兼顾性能和正确性。

通过上述优化,\sysname 的 RDMA 执行器能够在维持正确性(逻辑语义、通知语义、GPU 显存一致性)的前提下,显著减少 CPU-NIC MMIO 调用、降低 NIC 负载与 PCIe 带宽占用,从而在 GPU-to-GPU 大规模数据传输、异步高并发、多优先级调度场景中,获得接近 RDMA 和 NIC 硬件极限的性能表现。

\subsection{小结}

本节介绍了 \sysname 在 RDMA 链路上的完整执行路径,涵盖:

\begin{enumerate}
    \item GPU MemoryRegion 注册机制
    \item GPU Direct RDMA 的零拷贝数据路径
    \item Receiver 驱动与 Ticket 控制面协议
    \item Sender 的双队列匹配与 WR 链表构建
    \item write\_with\_imm 的必要性与 imm 传参
    \item Flusher 设计与 GPU 的内存一致性问题
    \item 批量 WR、减少 SIGNALED 优化 NIC 负载
\end{enumerate}

RDMA 链路执行器的设计实现了高性能、低延迟的 GPU-to-GPU 数据传输,是 \sysname 在多节点推理场景中性能表现的重要基础。

% \section{NVLink 链路执行器设计(TODO)}
% \subsection{统一接口的兼容设计}
% \subsection{执行器实现方向}
% \subsection{扩展性讨论}

% \section{流控机制与可靠性设计}

% \section{资源管理与内存模型}

% \section{本章小结}